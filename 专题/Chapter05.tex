\chapter{级数}
\section{数项级数}
% 新的小节
    \subsection{一般的级数}
    \begin{definition}[级数]

        设\(a_1,a_2,\cdots,a_n,\cdots\)是无穷多可列个实数,我们称形式和\[\sum_{k=1}^{\infty}a_k=a_1+a_2+\cdots+a_n+\cdots\]为无穷级数,称\(a_n\)为级数的通项或一般项,\enspace\(S_n=\sum\limits_{k=1}^{n}a_k\)为级数的第\(n\)个部分和。
        如果部分和\(S_n\)的极限存在且收敛于有限数\(S\),则称级数\(\sum\limits_{k=1}^{\infty}a_k\)收敛,记作\(\sum\limits_{k=1}^{\infty}a_k=S\),且称它的和为\(S\),否则称级数\(\sum\limits_{k=1}^{\infty}a_k\)发散。
    \end{definition}

    需要注意的是,级数的敛散性和其有限项的值无关。从某种意义上而言,级数的敛散性本质上就是数列的敛散性,所以根据数列极限的性质还可以得到:
    \begin{property}
        \begin{enumerate}
            \item 级数收敛的必要条件:如果级数\(\sum\limits_{n=1}^{\infty}\)收敛,那么其通项满足\(\lim\limits_{n\to\infty}a_n=0\)。
            \item 级数收敛的充要条件(Cauchy准则):级数\(\sum\limits_{n=1}^{\infty}\)收敛\(\iff\)任给\(\epsilon>0\),存在\(N=N(\epsilon)\),当\(m>n>N\)时,有\(\vert a_{n+1}+\cdots+a_m\vert=\left|\sum\limits_{k=n+1}^{m}a_k\right|<\epsilon\)。
            \item (线性性)设级数\(\sum\limits_{n=1}^{\infty}a_n\)和\(\sum\limits_{n=1}^{\infty}b_n\)都收敛,\(\lambda,\mu\in\mathbb{R}\),则级数\(\sum\limits_{n=1}^{\infty}(\lambda a_n+\mu b_n)\)也收敛,且\[\sum_{n=1}^{\infty}(\lambda a_n+\mu b_n)=\lambda\sum_{n=1}^{\infty}a_n+\mu\sum_{n=1}^{\infty}b_n.\]
        \end{enumerate}
    \end{property}

    回忆上一章利用黎曼和解决的很多问题,我们似乎可以把无限和与积分联系起来,下面的定理就将级数求和与广义积分联系了起来:
    \begin{lemma}
        设\(\{a_n\}\)是一列实数,在\([1,+\infty)\)中定义函数\(a(x)\)如下:设\(k\geq1\),当\(x\in[k,k+1)\)时,\(a(x)=a_k\),则级数\(\sum\limits_{n=1}^{\infty}a_n\)收敛当且仅当无穷积分\(\displaystyle\int_{1}^{+\infty}a(x)\mathrm{d}x\)收敛,且收敛的时候级数的和等于无穷积分的值。
    \end{lemma}\

    利用上面的引理,我们可以很轻松地把广义积分敛散性的判别法搬到级数上来,对于这种方法,我们不赘述证明,下面内容会按照传统的方式给出证明。

    对于一般的级数,下面的Dirichlet判别法和Abel判别法是核心。其证明的关键在于数列的\textbf{Abel变换}和\textbf{Abel引理}:
    \begin{lemma}
        \begin{enumerate}
            \item (Abel变换)设\(\{a_n\},\{b_n\}\)是两列实数,\(B_i=\sum\limits_{k=1}^{i}b_k\),则\[\sum_{i=1}^{n}a_ib_i=\sum_{i=1}^{n-1}(a_i-a_{i+1})B_i+a_nB_n.\]
            \item (Abel引理)设\(\{a_n\},\{b_n\}\)是两列实数,\(B_i=\sum\limits_{k=1}^{i}b_k\),若\(\{a_n\}\)为单调数列,\(\{B_n\}\)是有界数列,\(M=\sup\limits_{1\leq k\leq n}\vert B_k\vert\),则\[\sum_{i=1}^{n}a_ib_i\leq M(\vert a_1\vert+2\vert a_n\vert)\]
        \end{enumerate}
    \end{lemma}

    乍看Abel变换,你可能一时间摸不到头脑,其实Abel变换只是对\(\{a_n\},\{b_n\}\)分别进行了差分和逆差分,在这个想法下,我们只需要特别关注最后的\(a_nB_n\)项则可,而整个证明过程也只不过是裂项相消后对\(B_i\)求和。Abel引理的证明则体现了Abel变换的强大之处:在经历过差分和逆差分之后,得到的\(B_i\)一般都可以使用题目条件来控制,

    Dirichlet判别法和Abel判别法仅仅是Abel引理的直接推论,而Leibniz判别法仅仅是Dirichlet判别法的简单推论,我们在此一并列举出来:

    \begin{theorem}
        \begin{enumerate}
            \item (Dirichlet判别法)
            \item (Abel判别法)
            \item (Leibniz判别法)
        \end{enumerate}
    \end{theorem}
    % 新的小节
    \subsection{正项级数}
    \begin{theorem}[基本判别法]
        
    \end{theorem}
    \begin{theorem}[比较判别法]
        
    \end{theorem}
    \begin{theorem}[积分判别法]
        
    \end{theorem}
    \begin{theorem}[Kummer判别法]
        
    \end{theorem}

    {\heiti 注}(1)和前面一样,我们仍然可以通过求极限去寻找\(\lambda\),设\[\lim\limits_{n\to\infty}\left(\frac{1}{b_n}\frac{a_n}{a_{n+1}}-\frac{1}{b_{n+1}}\right)=\lambda,\]则当\(\lambda>0\)时,\(\displaystyle\sum_{n=1}^{\infty}a_n\)收敛;当\(\lambda<0\)时,且\(\displaystyle\sum_{n=1}^{\infty}b_n\)发散时,\(\displaystyle\sum_{n=1}^{\infty}a_n\)发散。

    (2)取\(b_n=1\),从Kummer判别法就直接可以得到d'Alembert判别法。
    \begin{corollary}
        \begin{enumerate}
            \item (Raabe判别法)
            \item (Gauss判别法)
        \end{enumerate}
    \end{corollary}

    各位对下面的Cauchy凝聚判别法可能感到有些莫名其妙,但是它的思想只不过是证明无穷级数发散的方法的一个变形,只需要胆大心细的分组放缩就可以证明。
    \begin{theorem}[Cauchy凝聚判别法]
        
    \end{theorem}
    % 新的小节
    \subsection{无穷乘积}
    \begin{definition}[无穷乘积]\label{def:数项级数:无穷乘积}
        
    \end{definition}

    % 新的小节
    \subsection{级数的乘积}
    有限个数的和的乘积,在我们眼中只是“转换了求和方式”,也就是使用了求和符号的魔法。举个简单的例子:\[\sum_{i=0}^{m}\sum_{j=0}^{n}a_ib_j=(a_0+a_1+\cdots+a_m)(b_0+b_1+\cdots+b_n)=\sum_{k=0}^{m+n}\sum_{i+j=k}a_ib_j=\sum_{k=0}^{m+n}c_k.\]
    \begin{definition}[Cauchy乘积]
        
    \end{definition}
    \begin{theorem}[Cauchy]\label{thm:数项级数:Cauchy定理}
        如果\(\sum\limits_{n=0}^{\infty}a_n\)和\(\sum\limits_{n=0}^{\infty}b_n\)都绝对收敛,则他们的乘积级数也绝对收敛,且\[\sum_{n=0}^{\infty}c_n=\left(\sum_{n=0}^{\infty}a_n\right)\left(\sum_{n=0}^{\infty}b_n\right).\]
    \end{theorem}

    如果将Cauchy定理的条件减弱,这时候下面的结果依然成立。
    \begin{theorem}[Mertens]\label{thm:数项级数:Mertens定理}
        如果\(\sum\limits_{n=0}^{\infty}a_n\)和\(\sum\limits_{n=0}^{\infty}b_n\)都收敛,且其中至少一个绝对收敛,则他们的乘积级数也收敛。\[\sum_{n=0}^{\infty}c_n=\left(\sum_{n=0}^{\infty}a_n\right)\left(\sum_{n=0}^{\infty}b_n\right).\]
    \end{theorem}

    衡量一个定理的强弱,可以从两方面来看:一般来说,定理的条件越弱,定理越强;结果越强,定理显然也越强,\enspace \autoref{thm:数项级数:Mertens定理}的条件已经足够弱了,我们只需要其中一个级数绝对收敛,就可以得到其乘积级数也收敛,并且这边绝对收敛的条件不能去掉,比如取\(a_n\)和\(b_n\)均为交错级数\((-1)^{n-1}\dfrac{1}{\sqrt{n}}\),那么所得到的乘积级数就是发散的。但是,当我们去掉绝对收敛的条件,并且加上其乘积级数收敛的条件,我们就能知道其值一定等于两个级数和的乘积,这就是下面的{\heiti \textbf{Abel}定理}。

    \begin{theorem}[Abel定理]
        设级数\(\sum\limits_{n=0}^{\infty}a_n\),\enspace \(\sum\limits_{n=0}^{\infty}b_n\)收敛,以及他们的乘积\(\sum\limits_{n=0}^{\infty}c_n\)也收敛,那么\[\sum_{n=0}^{\infty}c_n = \left(\sum_{n = 0}^{\infty}a_n\right)\left(\sum_{n = 0}^{\infty}b_n\right).\]
    \end{theorem}

    证明上面的定理,我们需要下面的\textbf{Abel}{\heiti 引理},Abel引理为我们提供了一种对发散级数进行求和的想法,我们在下一节中就会讲到。

    \begin{lemma}[Abel引理]\label{lem:数项级数:Abel引理}
        如果级数\(\sum\limits_{n=0}^{\infty}a_n\)收敛,那么\[\lim_{x\to 1^-}\sum_{n=0}^{\infty}a_nx^n = \sum_{n=0}^{\infty}a_n.\]
    \end{lemma}
    \begin{proof}
        
    \end{proof}

    有了\autoref{lem:数项级数:Abel引理},\enspace Abel定理就几乎显然了:对\(x\in(0,1)\),级数\(\sum\limits_{n=0}^{\infty}a_nx^n\)和\(\sum\limits_{n=0}^{\infty}b_nx^n\)都是绝对收敛的,他们的乘积级数\(\sum\limits_{n=0}^{\infty}c_nx^n\)也绝对收敛,根据\autoref{thm:数项级数:Cauchy定理},有\[\sum_{n=0}^{\infty}c_nx^n=\left(\sum_{n=0}^{\infty}a_nx^n\right)\left(\sum_{n=0}^{\infty}a_nx^n\right).\]
    令\(x\to1^-\),根据\autoref{lem:数项级数:Abel引理},我们就得到了Abel定理。

% 新的一节
\section{对数项级数的进一步讨论}
    \subsection{无穷乘积}
    \begin{definition}[无穷乘积]%\label{def:数项级数:无穷乘积}
        设\({p_n}\)是一列实数,我们将形式乘积\[\prod_{n=1}^{\infty}p_n=p_1p_2\cdots p_n\cdots\]称为{\heiti 无穷乘积},称\(P_n = p_1p_2\cdots p_n\)为无穷乘积的第\(n\)个{\heiti 部分积},如果部分积数列\({P_n}\)的极限存在,且极限为实数或者正负无穷,我们称此极限为无穷乘积的值,记为\[\prod_{n = 1}^{\infty}p_n = \lim_{n\to\infty}P_n.\]
        当此极限为非零实数的时候,称这个无穷乘积是{\heiti 收敛}的,否则称它是{\heiti 发散}的。
    \end{definition}

    如果某个$p_n$是零,显然无穷乘积的值就是零,下面我们假设每个$p_n$都是非零的,若无穷乘积$\prod\limits_{n=1}^{\infty}p_n$收敛于\(P\),则\[\lim_{n\to \infty}p_n = =\lim_{n\to\infty}\frac{P_n}{P_{n-1}} = \frac{P}{P} = 1.\]
    特别地,当\(n\)充分大的时候,必有\(p_n>0\),并且由于\[P_n = \prod_{k=1}^{n}p_k = \exp\left(\sum\limits_{k=1}^{n}\ln p_k\right)\]我们可以将无穷乘积化为无穷级数加以讨论,我们有:

    \begin{theorem}
        设\(p_n\)均大于零,记\(p_n = a_n +1\),则
        \begin{enumerate}
            \item 无穷乘积\(\prod\limits_{n=1}^{\infty}p_n\)收敛的充要条件是级数\(\sum\limits_{n=1}^{\infty}\ln p_n\)收敛,且\[\prod\limits_{n=1}^{\infty}p_n = \exp\left(\sum_{n=1}^{\infty}\ln p_n\right);\]
            \item 如果\(n\)充分大的时候\(a_n\)不变号,则无穷乘积\(\prod\limits_{n=1}^{\infty}p_n\)收敛的充要条件是级数\(\sum\limits_{n=1}^{\infty}a_n\)收敛;
            \item 如果级数\(\sum\limits_{n = 1}^{\infty}a_n\)和\(\sum\limits_{n = 1}^{\infty}a^2_n\)都收敛,则无穷乘积\(\prod\limits_{n=1}^{\infty}p_n\)也收敛。
        \end{enumerate}
    \end{theorem}

    这些的证明都很基本,使用\(\ln (1+x)\sim x\enspace(x\to0)\)以及\(\ln (1+x)\sim x-\dfrac{x^2}{2}\enspace(x\to0)\)与比较判别法即可得到。

    我们已经知道了,很多函数可以用级数的方式来表示,无穷乘积的美妙之处就在于,我们也可以将这些函数拿无穷乘积表示,这里以\(\sin x\)、\(\sinh x\)以及Riemann-zeta函数\(\zeta(x)\)为例。
    \subsection{交换求和顺序:级数的重排}
    \begin{theorem}[Riemann]
        如果\(\sum\limits_{n=1}^{\infty}a_n\)为条件收敛的级数,则可以将其重排为一个收敛级数,使得重排后的级数的和为任意指定的实数。
    \end{theorem}
    \begin{proof}
        
    \end{proof}
    \subsection{级数求和与求极限的可交换性}
    
    级数的和是其部分和的极限,也就是一个数列极限,那么我们考虑这样的问题:如果有一列数项级数,他们的和是另一列数,这列数的极限有什么性质?所以我们考虑依赖于双指标\(i,j\)的实数列\({a_{ij}}\).
    \begin{definition}[级数的一致收敛]
        一列收敛级数\(\sum\limits_{j = 1}^{\infty}=\alpha_i\)关于\(i\)一致收敛是指:任给\(\varepsilon>0\),存在与\(i\)无关的正整数\(N = N(\epsilon)\),当\(n>N\)时,有\[\left|\sum_{j=1}^{\infty}a_{ij}-\alpha_i\right|<\varepsilon.\]
    \end{definition}
    
    下面的定理给出了求极限和求和可交换次序的一个充分条件。
    \begin{theorem}
        设一列级数\(\sum\limits_{j=1}^{\infty}a_{ij}=\alpha_i\)关于\(i\)一致收敛,当\(j\geq1\)时,有\(\lim\limits_{i\to\infty}a_{ij}=a_j\),则极限\(\lim\limits_{i\to\infty}\alpha_i\)存在,级数\(\sum\limits_{j=1}^{\infty}a_j\)收敛,且\[\lim_{i\to\infty}\alpha_i = \sum_{j=i}^{\infty}a_j\enspace\text{或}\enspace\lim_{i\to\infty} \sum_{j=1}^{\infty}a_{ij} = \sum_{j=1}^{\infty}\lim_{n\to\infty}a_{ij}.\]
    \end{theorem}
    \subsection{Abel求和与Ces\`{a}ro求和}
    \begin{definition}[Abel求和]
        
    \end{definition}

    我们知道,级数\(\sum\limits_{n=0}^{\infty}(-1)^n\)在通常意义下是发散的,但是它在Abel意义下就可以求和了:
    \begin{definition}[Ces\`{a}ro求和]

    \end{definition}
    Ces\`{a}ro可和比Abel求和更强一些,也就是如果某个级数是Ces\`{a}ro可和的,那么它一定是Abel可和的,且两种意义下的和相等,这就是下面的定理。
    \begin{theorem}
        
    \end{theorem}
% 新的一节
\section{函数项级数}
\section{幂级数}
\section{求和与求导、积分的可交换性}
\section{习题:数项级数}
\begin{exercise}[判断下列正项级数的敛散性]
    \begin{enumerate}
        \item \(\displaystyle\sum\limits_{n=1}^{\infty}\left[\dfrac{1}{n}-\ln(1+\dfrac{1}{n})\right]\);
        \item \(x>0\),\enspace  \(\displaystyle\sum\limits_{n=1}^{\infty}n!(\frac{x}{n})^n\);
        \item 
    \end{enumerate}
\end{exercise}
\section{习题:函数项级数}