\chapter{函数极限与连续性}
	\section{Review:基本定义和定理}
	\subsection{函数极限}
	\begin{definition}[函数极限:$\epsilon-\delta$语言]
		设函数$y=f(x)$在点$x_0$的某个去心邻域中有定义,即存在$\rho>0$,使$$O(x_0,\rho) \backslash \{x_0\}\subset D_{f}$$如果存在实数$A$使得对于任意给定的$\epsilon>0$,可以找到$\delta>0$,使得当$0<|x-x_{0}|<\delta$时,成立$$|f(x)-f(x_0)|<\epsilon$$
		则称$A$是函数$f(x)$在$x_0$点的极限,记作$\lim\limits_{x\rightarrow x_0}f(x)=A$.\\
		如果不存在满足上述性质的常数$A$,则函数在$x_0$点的极限不存在.
	\end{definition}

	数列极限和函数极限同为极限,下面的Heine定理建立起了函数极限和数列极限之间的关系,它能将函数值数列的极限{\heiti 归结到}函数的极限上,所以我们一般将Heine定理称为Heien归结原理.

	\begin{theorem}[Heine归结原理]
		$\lim\limits_{x\rightarrow x_0}f(x)=A$的充分必要条件是:对于任意满足条件$\lim\limits_{n\rightarrow\infty}x_n=x_0,x_n\neq x_0\ (n=1,2,3,\cdots)$的数列$\{x_n\}$,
		相应的函数值数列$\{f(x_n)\}$成立:$$\lim\limits_{x\rightarrow\infty}f(x_n)=A.$$
	\end{theorem}

	事实上,当我们只需要判断函数在某一点(如$x_0$)的敛散性时,我们可以使用比较"宽松"一点的Heine定理:\par
	
	\begin{theorem}
		函数极限$\lim\limits_{x\rightarrow x_0}f(x)$存在的充分必要条件是:对于任意满足条件$\lim\limits_{n\rightarrow\infty}x_n=x_0$且$x_n\neq x_0(n=1,2,3,\cdots)$的数列$\{x_n\}$,相应的函数值数列$\{f(x_n)\}$收敛.
	\end{theorem}

	和数列极限相似,通过Heine定理,我们可以证明函数极限下的Cauchy收敛原理:
	
	\begin{theorem}[Cauchy收敛原理]
		函数极限$\lim\limits_{x\rightarrow x_0}f(x)$存在且有限的充分必要条件是:对于任意给定的$\epsilon>0$,存在$\delta>0$,使得对于任意满足条件$|x_{1}-x_{0}|<\delta,|x_{2}-x_{0}|<\delta$的$x_{1},x_{2}$,都有:$$|f(x_{1})-f(x_{2})|<\epsilon$$
	\end{theorem}
	\begin{property}[函数极限的性质]
		\begin{enumerate}
			\item (极限的唯一性)设$A$和$B$都是函数$f(x)$在$x_0$处的极限,则有$A=B$.
			\item (局部保序性)若$\lim\limits_{x\rightarrow x_0}f(x)=A,\lim\limits_{x\rightarrow x_0}g(x)=B$,
			且$A<B$,则存在$\delta>0$,当$0<|x-x_0|<\delta$时,成立$$f(x)>g(x).$$
			\item (局部有界性)若$\lim\limits_{x\rightarrow x_0}f(x)=A$,则存在$\delta>0$,使得$f(x)$在$O(x_0,\delta)\backslash\{x_0\}$中有界.
			\item (夹逼定理)若存在$r>0$,使得当$0<|x-x_0|<r$时,成立:$$g(x)\leq f(x)\leq h(x),$$且$\lim\limits_{x\rightarrow x_0}f(x)=\lim\limits_{x\rightarrow x_0}h(x)=A$,则$\lim\limits_{x\rightarrow x_0}f(x)=A$.
			\item (四则运算)设$\lim\limits_{x\rightarrow x_0}f(x)=A,\lim\limits_{x\rightarrow x_0}g(x)=B$,则:
					\begin{enumerate}
						\item $\lim\limits_{x\rightarrow x_0}(\alpha f(x)+\beta g(x))=\alpha A+\beta B;$
						\item $\lim\limits_{x\rightarrow x_0}(f(x)g(x))=AB;$
						\item $\lim\limits_{x\rightarrow x_0}\frac{f(x)}{g(x)}=\frac{A}{B}\ (B\neq0).$
					\end{enumerate}
		\end{enumerate}
	\end{property}
	\begin{definition}[单侧极限]
		设函数$f(x)$在$(x_0-\rho,x_0)$有定义$(\rho>0)$.如果存在实数$B$,对于任意给定的$\epsilon>0$,可以找到$\delta>0$,使得当$-\delta<x-x_0<0$时,成立$$|f(x)-B|<\epsilon,$$则称$B$是函数$f(x)$在点$x_0$处的\textbf{左极限},记为$$\lim\limits_{x\rightarrow x_0^{-}}f(x)=f(x_0^{-})=B.$$
		类似地,如果函数$f(x)$在$(x_0,x_0+\rho)$有定义$(\rho>0)$.如果存在实数$C$,对于任意给定的$\epsilon>0$,可以找到$\delta>0$,使得当$0<x-x_0<\delta$时,成立$$|f(x)-C|<\epsilon,$$则称$C$是函数$f(x)$在点$x_0$处的\textbf{右极限},记为$$\lim\limits_{x\rightarrow x_0^{+}}f(x)=f(x_0^{+})=C.$$
	\end{definition}

	事实上,自变量的极限过程可以分为六种情况:$x\rightarrow x_0,x\rightarrow x_0^{-},x\rightarrow x_0^{+},x\rightarrow\infty,x\rightarrow-\infty,x\rightarrow+\infty$.函数值的极限有四种情况:$f(x)\rightarrow A,f(x)\rightarrow\infty,f(x)\rightarrow-\infty,f(x)\rightarrow+\infty$.
	这样,我们就可以将函数极限的定义扩充到很宽的情况.
	\begin{proposition}[单调函数单侧极限存在定理]
		设$f(x)$在区间$(a,b)$上单调,则$\lim\limits_{x\rightarrow b^{-}}f(x)=f(b^{-})$一定有意义,即若函数$f(x)$单调增加时,如$f(x)$在$(a,b)$有上界,则$f(b^{-})=\sup\left\{y\vert\ \exists\ x\in(a,b),y=f(x)\right\}$,否则$f(b^{-})=+\infty$,反之亦然.
	\end{proposition}
	\subsection{函数的连续性}
	\begin{definition}[连续函数]
		设函数$f(x)$在点$x_0$的某个去心邻域中有定义,并且成立$$\lim\limits_{x\rightarrow x_0}=f(x_0),$$
		则称函数$f(x)$在\textbf{点}$\bm{x_0}$\textbf{连续},而称$x_0$是函数$f(x)$的连续点.\\
		如果函数$f(x)$在区间$(a,b)$的每一点都连续,则称函数$f(x)$在\textbf{开区间}$(a,b)$上\textbf{连续}.
	\end{definition}
	\begin{definition}[振幅]
		设$f(x)$在$x_0$的一个开邻域内有定义,称$$\omega_{f}(x_0,r)=\sup\{|f(x')-f(x'')|\vert x',x''\in(x_0-r,x_0+r)\}\ (r>0)$$
		为$f$在区间$(x_0-r,x_0+r)$上的\textbf{振幅}.显然,$\omega_f(x_0,r)$关于$r\rightarrow0^{+}$单调递减,因此$$\omega_f(x_0)=\lim\limits_{r\rightarrow0^{+}}\omega_f(x_0,r)$$
		存在(不一定有限),称为$f$在$x_0$处的\textbf{振幅}.\\
		为了用振幅来刻画一致连续性,设$f$定义在区间$I$中,$r>0$.令$$\omega_f(r)=\sup\{|f(x')-f(x'')|\vert x',x''\in I,|x'-x''|<r\},$$
		则$\omega_f(r)$关于$r\rightarrow0^{+}$单调递减.
	\end{definition}
	\begin{theorem}
		函数$f(x)$在$x_0$处连续当且仅当$\omega_f(x_0)=0$.\\
		函数$f(x)$在$I$中一致连续当且仅当$\lim\limits_{r\rightarrow0^{+}}\omega_f(r)=0$.
	\end{theorem}

	这种用振幅来刻画函数的连续性的想法也出现在刻画函数的可积性上.
	\begin{definition}[三类不连续点]
		\begin{enumerate}
			\item 第一类不连续点:函数$f(x)$在$x_0$左、右极限都存在但是不相等,即$f(x^{+})\neq f(x^{-})$。第一类不连续点又称为跳跃点,右极限和左极限之差$f(x^{+})-f(x^{-})$称为函数$f(x)$在$x_0$处的跃度;
			\item 第二类不连续点:函数$f(x)$在$x_0$的左、右极限中至少一个不存在;
			\item 第三类不连续点:函数$f(x)$在$x_0$的左、右极限都存在且相等,但是不等于$f(x_0)$或者函数$f(x)$在$x_0$处无定义。第三类不连续点又称为可去不连续点。
		\end{enumerate}
	\end{definition}
	\begin{property}
		在区间$(a,b)$上的单调函数的不连续点是第一类不连续点,即左右极限都存在但不相等.
	\end{property}
	\begin{proof}
		设$x_0\in(a,b)$是任意一点,集合$\left\{f(x)\vert x\in(a,x_0)\right\}$非空且有上界,则一定存在上确界$\alpha$,$$\alpha=\sup\left\{f(x)\vert x\in(a,x_0)\right\}.$$
		对一切$x\in(a,x_0),f(x)\leq\alpha$,且有$\forall\epsilon>0$,$\exists\ x^{'}\in(a,x_0)$,使得$\alpha-f(x^{'})<\epsilon$,取$\delta=x_0-x^{'}$,则当$-\delta<x-x_0<0$时,$-\epsilon<f(x^{'})-f(x_0)\leq f(x)-f(x_0)\leq0$,
		这样有$\lim\limits_{x\rightarrow x_{0^{-}}}f(x)=\alpha$,同理$\lim\limits_{x\rightarrow x_{0^{+}}}f(x)=\beta$,其中$\beta=\inf\left\{f(x)\vert x\in(x_0,b)\right\}.$

	\end{proof}
	\begin{theorem}[连续函数的四则运算]
		
	\end{theorem}
	\begin{theorem}[反函数的存在性]
		若函数$y=f(x),x\in D_f$是单调的,则它的反函数$x=f^{-1}(y),y\in R_f$存在,而且$f^{-1}(y)$的单调性和$f(x)$相同.
	\end{theorem}
	\begin{theorem}[反函数的连续性]
		设函数$f(x)$在闭区间$\left[a,b\right]$连续且严格单调增加,$f(a)=\alpha ,f(b)=\beta$,则它的反函数$f^{-1}(y)$在$\left[\alpha ,\beta\right]$连续且严格单调增加。
	\end{theorem}
	\begin{theorem}[复合函数的连续性]
		若$u=g(x)$在点$x_0$连续,$g(x_0)=u_0$,又$y=f(u)$在点$u_0$连续,则复合函数$y=f\circ g(x)$在点$x_0$连续。
	\end{theorem}
	\begin{theorem}[初等函数的连续性]
		我们认为{\kaishu 初等函数}是指的是六类{\kaishu 基本初等函数}经过{\kaishu 有限次}四则运算和复合运算所产生的函数.\\
		初等函数在其{\kaishu 定义区间}上连续.
	\end{theorem}

	我们利用实数系的完备性定理可以证明以下连续函数的性质,
	\begin{theorem}[闭区间上连续函数的性质]
		\begin{enumerate}
			\item (有界性定理)
			\item (最值定理)
			\item (介值定理)
			\item (零点存在定理)
		\end{enumerate}
	\end{theorem}
	\begin{corollary}
		\begin{enumerate}
			\item 设$f(x)$是$I=\left[a,b\right]$上的连续函数,则$f(I)=\left[m,M\right]$其中$m,M$分别是$f$在$\left[a,b\right]$上的最小值和最大值.
			\item 设$f(x)$是区间$I$上的连续函数,则$f(I)$也是一个区间(可以退化成一个点).
			\item 设$f(x)$是区间$I$上的连续函数,则$f(x)$可逆当且仅当$f(x)$是严格单调函数.
		\end{enumerate}
	\end{corollary}
	\begin{definition}[一致连续]
		
	\end{definition}
	\begin{theorem}[一致连续性的判断]
		设函数$f(x)$在区间$X$上有定义,则$f(x)$在$X$上一致连续的充分必要条件是: 对任何点列$\left\{x_{n}^{'}\right\}\ (x_{n}^{'}\in X)$和$\left\{x_{n}^{''}\right\}\ (x_{n}^{''}\in X)$,只要满足$\lim\limits_{n\rightarrow\infty}(x_{n}^{''}-x_{n}^{''})=0$,就成立$\lim\limits_{n\rightarrow\infty}(f(x_n^{'})-f(x_n^{''}))=0$.
	\end{theorem}
	\begin{proof}
		首先证明必要性:\\
		由于函数$f(x)$的一致连续性,$$\forall\epsilon>0,\exists\delta>0,\forall x^{'},x^{''}\in X,(\lvert x^{'}-x^{''}\rvert<\delta):\lvert f(x^{'})-f(x^{''})\rvert<\epsilon.$$
	\end{proof}
	\begin{theorem}[Contor定理]
		闭区间上的连续函数在此区间一致连续.\\
		换句话说:若函数$f(x)$在闭区间$\left[a,b\right]$上连续,则它在$\left[a,b\right]$上一致连续.
	\end{theorem}
	\begin{theorem}
		函数$f(x)$在{\kaishu 有限}开区间$(a,b)$上连续,则$f(x)$在$(a,b)$上一致连续的充分必要条件是:$f(a+)$与$f(b-)$存在.
	\end{theorem}
	\subsection{无穷大量和无穷小量的阶}
	\begin{definition}[无穷小量]
		
	\end{definition}
	\begin{definition}[无穷大量]
		
	\end{definition}
	\begin{definition}[等价量]
		
	\end{definition}
	\begin{theorem}[替换定理]
		设$f(x),g(x),f_1(x)$在$x_0$的某个去心邻域$U$中有定义,且$f(x)\thicksim f_1(x)\ (x\rightarrow x_0)$,那么:
		\begin{enumerate}
			\item 当$\lim\limits_{x\rightarrow x_0}f_1(x)g(x)=A$时,有$\lim\limits_{x\rightarrow x_0}f(x)g(x)=A$;
			\item 当$\lim\limits_{x\rightarrow x_0}\frac{g(x)}{f_1(x)}=A$时,有$\lim\limits_{x\rightarrow x_0}\frac{g(x)}{f(x)}=A$.
		\end{enumerate}
	\end{theorem}
	\begin{example}[常见的等价量]
		\begin{enumerate}
			\item $\ln(1+x)\thicksim x\ (x\rightarrow0)$
			\item $\sin x\thicksim x\ (x\rightarrow0)$
			\item $\cos x\thicksim 1-\frac{x^2}{2}\ (x\rightarrow0)$
			\item $e^x\thicksim1+x\ (x\rightarrow0)$
		\end{enumerate}
	\end{example}
	\subsection{例题与练习}
	\begin{example}[函数极限的换元法]
		设$\lim\limits_{x\rightarrow a}g(x)=A,\lim\limits_{y\rightarrow A}f(y)=B$成立,且在点$a$的某个邻域上$g(x)=y$.如果满足以下条件之一:
		\begin{enumerate}
			\item 存在点$a$的一个去心邻域$O_{\delta_0}(a)-{a}$,在其中$g(x)\neq A$;
			\item $\lim\limits_{y\rightarrow A}f(y)= f(A)$;
			\item $A=\infty$,且$\lim\limits_{x\rightarrow A}f(y)$有意义;
		\end{enumerate}
		则成立$$\lim\limits_{x\rightarrow a}f(g(x))=\lim\limits_{y\rightarrow A}f(y)=B.$$
	\end{example}
	\begin{example}
		设$\lim\limits_{x\rightarrow a}g(x)=A,\lim\limits_{y\rightarrow A}f(y)=B$,证明:极限$\lim\limits_{x\rightarrow a}f(g(x))$只有三种可能:\\
		(1):\ $\lim\limits_{x\rightarrow a}f(g(x))=B$;\ \ (2):\ $\lim\limits_{x\rightarrow a}f(g(x))=f(A)$;\ \ (3):\ 极限$\lim\limits_{x\rightarrow a}f(g(x))$不存在.
	\end{example}
	\begin{example}
		设$f(x)$是定义在区间$I$上的函数,如果存在$0<\alpha\leq1$,以及常数$M$,使得$$|f(x_1)-f(x_2)|\leq M|x_1-x_2|^{\alpha},\ \forall x_1,x_2\in I$$
		则称$f(x)$是$I$中的$\alpha$阶Holder函数,当$\alpha=1$时也称为Lipschitz函数.\\
		证明:Holder函数都是\textbf{一致连续的}.
	\end{example}
	\begin{example}[Brouwer不动点定理]
		函数$f:\left[0,1\right]\rightarrow\left[0,1\right]$是连续映射,那么$f$有不动点,即存在$x\in\left[0,1\right]$,使得$f(x)=x$.
	\end{example}
	\begin{example}
		证明:不存在函数$f:\mathbb{R}\rightarrow\mathbb{R}$在所有无理点不连续,而在所有有理点连续.
	\end{example}

	\section{$e$和$e^x$的构造}
	\begin{proposition}
		证明:数列$\{(1+\frac{1}{n})^n\}$单调递增,数列$\{(1+\frac{1}{n})^{n+1}\}$单调递减.
	\end{proposition}
	\begin{theorem}[e的构造]
		数列$\{(1+\frac{1}{n})^n\}$和数列$\{(1+\frac{1}{n})^{n+1}\}$收敛于相同的极限,记此极限为$e$.
	\end{theorem}
	\begin{example}
		证明$$\lim\limits_{n\rightarrow\infty}\sum_{k=0}^{n}\frac{1}{k!}=e$$
	\end{example}
	\begin{example}
		记$b_n=1+\frac{1}{2}+\frac{1}{3}+\cdots+\frac{1}{n}-\ln n$,证明:数列$\{b_n\}$收敛.
	\end{example}
	\begin{example}
		证明:$$\lim\limits_{n\rightarrow\infty}\left[1-\frac{1}{2}+\frac{1}{3}-\cdots+(-1)^{n+1}\frac{1}{n}\right]=\ln2.$$
	\end{example}
	\begin{proof}
		$$\left(\frac{p}{q}\right)=\left\{\begin{array}{rcl}1&if\ n^2\equiv q(mod p) for\ integer\ r\\-1&otherwise\end{array}\right.$$
	\end{proof}