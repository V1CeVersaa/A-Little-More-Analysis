\chapter{曲线积分和曲面积分}
\section{曲线和曲面}
    \subsection{曲面的第一基本形式}
    一个曲面\(\mit\Omega\)的参数方程为\[\bm{r}=\bm{r}(u,v),\ (u,v)\in D\]
    我们假定\(\bm{r}=\bm{r}(u,v)\)连续可微足够多次并且满足正则条件:\(\bm{r}_u\times\bm{r}_v\neq0,\ \forall(u,v)\in D\).考察曲面上的一条曲线\(L: \bm{r}=\bm{r}(u(t),v(t))\),\ \(t\in J\),且\(u(t)\)和\(v(t)\)都连续可微.上式对\(t\)求导得:
    \[\bm{r}'=\bm{r}_u\frac{\mathrm{d}u}{\dd t}+\bm{r}_u\frac{\dd u}{\dd t},\enspace\dd\bm{r}=\bm{r}_u\dd u+\bm{r}_v\dd v.\]
    曲线\(L\)的弧长微元可以表示为:\[\dd s=\Vert\bm{r}'\Vert\dd t=\pm\Vert\bm{r}'\dd t\Vert=\pm\Vert\dd\bm{r}\Vert.\]
    进而有:\[\dd s^2=\Vert\dd\bm{r}\Vert^2=\langle\dd\bm{r},\dd\bm{r}\rangle=E\dd u^2+2F\dd u\dd v+G\dd v^2.\]
    其中:\[E=\bm{r}_u^2,\enspace F=\bm{r}_u\cdot\bm{r}_v,\enspace G=\bm{r}_v^2.\]
    并且记\[I(\dd u,\dd v)=E\dd u^2+2F\dd u\dd v+G\dd v^2.\]
    于是曲面\(\mit\Omega\)上的曲线\(L\)的弧长可以按照这样计算:\[s=s_0+\int_{t_0}^{t}\sqrt{E\left(\frac{\dd u}{\dd t}\right)^2+2F\frac{\dd u}{\dd t}\frac{\dd v}{\dd t}+G\left(\frac{\dd v}{\dd t}\right)^2}\dd t=s_0+\int_{t_0}^{t}\sqrt{I\left(\frac{\dd u}{\dd t},\frac{\dd v}{\dd t}\right)}\dd t.\]

    于是我们将微分\(\dd u\)和\(\dd v\)的二次型\[I=E\dd u^2+2F\dd u\dd v+F\dd v^2\]称为曲面\(\mit\Omega\)的第一基本形式.曲面上的曲线的弧长取决于这个曲面的第一基本形式,后面将要见到,曲线块的面积也取决于这个曲面的第一基本形式.因而有:\textbf{曲面的第一基本形式决定了曲面的度量性质}.
    
\section{第一类曲线积分}
\section{第二类曲线积分}
\section{第一类曲面积分}
\section{第二类曲面积分}

    值得注意的是,为了书写的简便,我们有时候也会将\(\dd x\wedge\dd y,\dd y\wedge\dd z\)和\(\dd z\wedge\dd x\)等记号简写为\(\dd x\dd y,\dd y\dd z\)和\(\dd z\dd x\).比如积分\(\displaystyle\int_{\mit\Sigma}f(x,y,z)\dd x\wedge\dd y\)就可以简写为\(\displaystyle\int_{\mit\Sigma}f(x,y,z)\dd x\dd y\).
    \begin{example}
        计算积分\(\displaystyle\mit\Phi=\int_{\mit\Sigma}x^2\mathrm{d}y\wedge\mathrm{d}z+y^2\mathrm{d}z\wedge\mathrm{d}x+z^2\mathrm{d}x\wedge\mathrm{d}y\),其中\(\mit\Sigma\)是球面\((x-a)^2+(y-b)^2+(z-c)^2=R^2\),方向为外侧。
    \end{example}

\section{Green公式、Gauss公式和Stokes公式}
    \subsection{Green公式}
    \begin{example}
        设\(\mit\Omega\)为包含原点的有界区域,其边界为\(C^1\)曲线,方向为诱导定向,计算积分\[I=\int_{\partial\mit\Omega}\frac{-y\dd x}{x^2+y^2}+\frac{x\dd y}{x^2+y^2}.\]
    \end{example}
    \subsection{Gauss公式}
    \begin{theorem}[Gauss公式]
        
    \end{theorem}
    
    将以函数\(P,Q,R\)为分量的向量场\(X=(P,Q,R)\)的散度\(\nabla\cdot X\)(或\(\mathrm{div}X\))定义为\[\mathrm{div}X=\nabla\cdot X=\frac{\partial P}{\partial x}+\frac{\partial Q}{\partial y}+\frac{\partial R}{\partial z}.\]
    散度是将Del算符(\(\nabla\))点积一个向量场X,效果是将一个向量场转换成了一个标量场.利用散度,Gauss公式可以写为以下形式:\[\int_{\mit\Omega}\nabla\cdot X\dd x\dd y\dd z=\int_{\partial\mit\Omega}X\cdot\vec{n}\dd\sigma.\]
    其中\(\vec{n}\)是边界曲面的单位外法向量,上式也被称为散度定理。
    \subsection{Stokes公式}